\documentclass[a4paper, 12pt]{report}


\usepackage[english]{babel}
\usepackage[utf8]{inputenc}
\usepackage{graphicx}
\usepackage[a4paper, width=150mm, top=25mm, bottom=25mm]{geometry}
\usepackage{mathptmx}
\usepackage[T1]{fontenc}
\usepackage{enumitem}
\usepackage{amsmath}
\usepackage{index}
\usepackage{fancyhdr}
\usepackage{titlesec}
\usepackage{subfig}
\usepackage{listings}
\usepackage{listings-rust}
\titleformat{\chapter}{\normalfont\huge}{\thechapter.}{20pt}{\huge\it}

\graphicspath{ {images/} }
\makeindex

\usepackage{listings}
\usepackage{xcolor}

\definecolor{codegreen}{rgb}{0,0.6,0}
\definecolor{codegray}{rgb}{0.5,0.5,0.5}
\definecolor{codepurple}{rgb}{0.58,0,0.82}
\definecolor{backcolour}{rgb}{0.95,0.95,0.92}

\lstdefinestyle{mystyle}{
    backgroundcolor=\color{backcolour},   
    commentstyle=\color{codegreen},
    keywordstyle=\color{magenta},
    numberstyle=\tiny\color{codegray},
    stringstyle=\color{codepurple},
    basicstyle=\ttfamily\footnotesize,
    breakatwhitespace=false,         
    breaklines=true,                 
    captionpos=b,                    
    keepspaces=true,                 
    numbers=left,                    
    numbersep=5pt,                  
    showspaces=false,                
    showstringspaces=false,
    showtabs=false,                  
    tabsize=2
}


\usepackage{hyperref}
\hypersetup{
    colorlinks=true,
    linkcolor=blue,
    filecolor=magenta,      
    urlcolor=cyan,
}

\urlstyle{same}
\lstset{style=mystyle}

\makeatletter
\renewcommand{\bibname}{References}
\renewenvironment{thebibliography}[1]
     {\chapter{\bibname}%
      \@mkboth{\MakeUppercase\bibname}{\MakeUppercase\bibname}%
      \list{\@biblabel{\@arabic\c@enumiv}}%
           {\settowidth\labelwidth{\@biblabel{#1}}%
            \leftmargin\labelwidth
            \advance\leftmargin\labelsep
            \@openbib@code
            \usecounter{enumiv}%
            \let\p@enumiv\@empty
            \renewcommand\theenumiv{\@arabic\c@enumiv}}%
      \sloppy
      \clubpenalty4000
      \@clubpenalty \clubpenalty
      \widowpenalty4000%
      \sfcode`\.\@m}
     {\def\@noitemerr
       {\@latex@warning{Empty `thebibliography' environment}}%
      \endlist}
\makeatother


\begin{document}
\title{\Large{\textbf{Thesis Template}}}
\author{Henning W.}
\date{09/09/1999}
\maketitle
\tableofcontents
\setcounter{page}{2}
\fancyhf{}
\renewcommand{\headrulewidth}{2pt}
\renewcommand{\headrulewidth}{2pt}
\fancyhead{\leftmark}
\fancyfoot{\thepage}
\chapter{ÆØÅæøå}
\chapter{graphics}

\begin{figure}[ht]
\centering
\caption{cake}
\includegraphics[width=12cm]{cph_logo.png}
\caption{cake}
\label{trash}
\label{praise kek}
\end{figure}
ref:
\ref{trash}
pageref:
\pageref{trash}
\begin{figure}[!tbp]
  \centering
  \subfloat[trash.]{\includegraphics[width=0.4\textwidth]{cph_logo.png}\label{fig:f1}}
  \hfill
  \subfloat[garbage.]{\includegraphics[width=0.4\textwidth]{cph_logo.png}\label{fig:f2}}
  \caption{so much trash and garbage}
\end{figure}

\chapter{sections and stuff}

\section{Some numbered section}
\section*{Some non-numbered section}
\subsection{Some numbered Subsection}
\subsection*{Some non-numbered Subsection}
\subsubsection*{Some Subsubsection}
\paragraph{some paragraph}
\subparagraph{some subparagraph}

\chapter{lists}

\begin{itemize}
	\item{Item 1}
	\item{Item 2}
	\item{Item 3}
\end{itemize}
\renewcommand{\labelitemi}{$\spadesuit$}
\begin{itemize}
	\item{Item 1}
	\item{Item 2}
	\item{Item 3}
\end{itemize}
\begin{enumerate}
 \item Enumerated Item 1
 \item Enumerated Item 2
 \item Enumerated Item 3
\end{enumerate}
\begin{enumerate}[label=\roman*]
 \item Enumerated Item 1
 \item Enumerated Item 2
 \item Enumerated Item 3
\end{enumerate}

\chapter{tables}

\begin{table}[h!]
  \begin{center}
    \caption{Some table}
    \label{tab:table1}
    \begin{tabular}{l|c|r}
      \textbf{Value 1} & \textbf{Value 2} & \textbf{Value 3}\\
      $\alpha$ & $\beta$ & $\gamma$ \\
      \hline
      1 & 1110.1 & a\\
      2 & 10.1 & b\\
      3 & 23.113231 & c\\
      4 & 25.113231 & d\\
    \end{tabular}
  \end{center}
  \caption*{Description goes here}
\end{table}
table ref:
\ref{tab:table1}

\chapter{Code Listing}

\begin{lstlisting}[language=Python]
class SomePythonClass:
	def __init__(self, variable):
		self.variable = variable
\end{lstlisting}

\begin{lstlisting}[language=Rust]
struct SomeRustStruct {
	variable: String,
}
impl SomeRustStruct {
	fn print_variable(&self){
		println!("{}", self.variable);
	}
}
\end{lstlisting}

\begin{lstlisting}[language=java]
public class SomeJavaClass(){
	private String variable;

	public SomeJavaClass(String variable){
		this.variable = variable;
	}
	
	public String getVariable(){
		return this.variable
	}	
}
\end{lstlisting}

\chapter{Math Equations}

inline: $2+2=4$


\begin{flalign*}
	&2+2=4&\\
\end{flalign*}

$$ 2+2=4 $$

a: $$ (\frac{7}{8} )+ (\frac{ 0}{0}) = (\frac{7} { 8}) $$

3 \&\& 4: $$ \vec{a} + \vec{b} + \vec{c} = (\frac{3}{2} )+ (\frac{5}{1} )+ (\frac{-2}{6}) = (\frac{6}{9} )= (\frac{ni}{ce}) $$

$$ \hat{a} = (\frac{3}{0}) / \sqrt(3^2 + 0^2 ) = (\frac{3}{0}) / \sqrt(9) = (\frac{1}{0}) $$ $$ \hat{a} = (\frac{2}{2}) / \sqrt(2^2 + 2^2 ) = (\frac{2}{2}) / \sqrt(8) = (\frac{0.71}{0.71}) $$

$$ |\vec{a}| = \sqrt(2^2 + 3^2) = \sqrt(13) = 3.61 $$ $$ |\vec{b}| = \sqrt(4^2 + 6^2) = \sqrt(52) = 7.21 $$

$$ 2x3=6 $$

\begin{thebibliography}{9}
	\bibitem{}Sharma H., Dominic P., and Christine H. L., Stabilization of Methanol-Bases 		Suspensions, \emph{J. K. Ceram. Soc.}, 89, (2005) 450-485.
	\bibitem{}Sharma H., Dominic P., and Christine H. L., Stabilization of Methanol-Bases 		Suspensions, \emph{J. K. Ceram. Soc.}, 89, (2005) 450-485.
	\bibitem{}\href{https://chickenonaraft.com}{Chicken on a raft} To da monday mornin'.

\end{thebibliography}

\chapter{Todo}



\begin{enumerate}
 \item (Done) user creation or at least switch user, will be done via a route. 
    this is to demonstrate how some functionalities will be disabled after using a restricted user.
    This route will use the "usr.sql" script in the misc folder.
 \item (Created. Needs testing. Depends on 6) Cat and dog must have the view functionalities configured.

 \item (Done) Pop script. As part of the assignment description, the following entities must be created upon execution:

    2 cities
    2 vets
    10 general pets
    5 dogs
    5 cats
    10 caretakers

    Question is whether I'm going to create an .sql file or whether I'm just going to use the functions I've created.
    Indirectly I AM using sql ;)
    
	\item (Some chicken and egg paradox going on. Skipping due to time constraints... Program still dependant on Rust) Separate docker functionality for rust container. Project must be executable by people who doesn't want to go through the rust installation process.
    3.1 Database connection will be relying on the docker network instead of localhost.
        .env must have a separate option in some way. Going to be complicated.
	\item (Done) ER diagram
	\item (Done) Expanded CRUD for cat/dog ...
	\item Need better argument for inheritance strategy...
\end{enumerate}


Good testing tactic:
\begin{lstlisting}[language=Rust]
let mut conn = new::<Postgres>().await?;

let _ = sqlx::query!(r#"select pg_notify('chan', 'message')"#)
    .execute(&mut conn)
    .await?;
Ok(())

\end{lstlisting}


\end{document}